\section{Diskussion}
\label{sec:diskussion}

Aus der Differenz der Drehwinkel zwischen dotierten Proben und reiner Probe lässt sich in Abbildung~\ref{fig:masse} der
Beitrag der quasifreien Elektronen isolieren. Wie zuvor erwähnt, werden dabei einige Messwerte ausgelassen. Diese Entscheidung ist
hauptsächlich in der beobachteten Streuung begründet, die wahrscheinlich auf Verunreinigungen der Appartur und unzulängliche
Justierung der Bauteile zurückzuführen ist. Dazu kommt, dass die Nullkalibrierung teilweise durch Rauschen überdeckt wird,
sodass die Ergebnisse nicht mehr brauchbar sind. 

Aus dem auf diese Weise reduzierten Datensatz ergeben sich
\begin{align*}
    m^{*}_1 = \input{build/m-1.tex} && m^{*}_2 = \input{build/m-2.tex}
\end{align*}
für die effektiven Massen der Leitungselektronen. Im Vergleich mit einem Literaturwert $m^{*} = (\num{0.078(0.004)})\,m_0$
\cite{PhysRev.114.59} zeigt sich zwar eine ähnliche Größenordnung, die von uns aufgeführten Unsicherheiten scheinen aber
zu gering auszufallen. Um eine höhere Übereinstimmung zu erhalten, müssten systematische Einflüsse besser verstanden werden, die
aber nicht ohne weiteres zugänglich sind.

Allgemein fallen einige weitere systematische Fehlerquellen auf. Optische Elemente wie Linsen, Filter und Prismen haben teilweise
zerkratzte oder befleckte Oberflächen. Besonders für die Polarisatoren können schon geringfügige Fehleinstellungen für eine
leicht elliptische Polarisation sorgen und die Resultate entsprechend verfälschen. Auch die geometrische Anordnung des Aufbaus ist
fehleranfällig. So lässt sich etwa die Winkeleinstellung der Photowiderstände nicht vollständig fixieren. Zusätzlich wird der
senkrechte Lichteinfall auf Probe und Prismen mit bloßem Auge abgeschätzt. Selbiges gilt für das Ablesen der händisch
eingestellten Winkel am Goniometer.

Zuletzt sei noch die Beobachtung eines leicht abfallenden Spulenstroms erwähnt. Erklären lässt sich dieser mit einer Erwärmung
des Elektromagneten und einer resultierenden Abschwächung der Flussdichte. Die Annahme eines zeitlich völlig konstanten Feldes
ist somit strenggenommen nicht erfüllt, wobei die auftretende Variation im Vergleich zu den anderen Unsicherheiten
verschwindend gering sein dürfte.
