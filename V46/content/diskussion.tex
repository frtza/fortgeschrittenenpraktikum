\section{Diskussion}
\label{sec:diskussion}

Bei demn Versuch existieren viel systematische Fehlerquellen. Es ist beispielsweise aufgefallen, dass die Interferenzfilter oft 
mit Fettflecken beschmutzt waren. 
Außerdem hatte sich der Elektromagnet erwärmt und so wurde der Widerstand größer und das Magnetfeld war nicht vollständig zeitlich
konstant.
Dazu kommt, dass bei der Justierung schon kleine Fehleinstellungen das Ergebniss 
beinflusst haben können. Zum Beispiel wurde es auschließlich mit dem Auge abgeschätzt, ob das Licht der Halogenlampe nach der
Sammellinse exakt senkrecht auf das Prisma fällt.  Kontrolliert werden konnte dies zusätzlich nicht.
Auch zu beachten ist, dass die Drehwinkel $\theta_1$ und $\theta_2$ händisch eingestellt wurden und es zusätzlich
beim Ablesen zu Unsicherheiten kommen kann.