\section{Diskussion}
\label{sec:diskussion}

Aus der Differenz der Drehwinkel zwischen dotierten Proben und reiner Probe lässt sich in Abbildung~\ref{fig:masse} der
Beitrag der quasifreien Elektronen isolieren. Wie zuvor erwähnt, werden dabei einige Messwerte ausgelassen. Diese Entscheidung ist
hauptsächlich in der beobachteten Streuung begründet, die wahrscheinlich auf Verunreinigungen der Appartur und unzulängliche
Justierung der Bauteile zurückzuführen ist. Dazu kommt, dass die Nullkalibrierung teilweise durch Rauschen überdeckt wird,
sodass die Ergebnisse nicht mehr brauchbar sind. 

Aus dem auf diese Weise reduzierten Datensatz ergeben sich
\begin{align*}
    m^{*}_1 = \input{build/m-1.tex} && m^{*}_2 = \input{build/m-2.tex}
\end{align*}
für die effektiven Massen der Leitungselektronen. Der Vergleich mit einem Literaturwert $m^{*} = (\num{0.078(0.004)})\,m_0$
\cite{PhysRev.114.59} zeigt sich zwar eine ähnliche Größenordnung, die von uns aufgeführten Unsicherheiten scheinen aber
zu gering auszufallen. Um eine bessere Übereinstimmung zu erhalten, müssten systematische Einflüsser besser verstanden werden, die
aber so nicht zugänglich sind.


Bei demn Versuch existieren viel systematische Fehlerquellen. Es ist beispielsweise aufgefallen, dass die Interferenzfilter oft 
mit Fettflecken beschmutzt waren. 
Außerdem hatte sich der Elektromagnet erwärmt und so wurde der Widerstand größer und das Magnetfeld war nicht vollständig zeitlich
konstant.
Dazu kommt, dass bei der Justierung schon kleine Fehleinstellungen das Ergebniss 
beinflusst haben können. Zum Beispiel wurde es auschließlich mit dem Auge abgeschätzt, ob das Licht der Halogenlampe nach der
Sammellinse exakt senkrecht auf das Prisma fällt.  Kontrolliert werden konnte dies zusätzlich nicht.
Auch zu beachten ist, dass die Drehwinkel $\theta_1$ und $\theta_2$ händisch eingestellt wurden und es zusätzlich
beim Ablesen zu Unsicherheiten kommen kann.
