\section{Aufbau}
\label{sec:Aufbau}

Zur Bestimmung der effektiven Masse der Leitungselektronen von n-dotiertem Gallium Arsenid, wird die Faraday Rotation
gemessen. Dabei wird der in Abbildung \ref{fig:apparatur} dargestellte Versuchaufbau verwendet.

\begin{figure}[H]
    \centering
    \includegraphics[width=0.8\textwidth]{content/grafik/apparatur.pdf}
    \caption{Schematische Darstellung der Messapparatur. \cite{faraday}}
    \label{fig:apparatur}
\end{figure}

Die Halogenlampe besitzt ein Emissionsspektrum im hauptsächlich Infrarotbereich. Dieses Licht trifft auf eine
Sammellinse, sodass das Licht parallelisiert und anschließend durch einen Licht-Zerhacker gepulst wird.
Das nun gepulste Licht trifft senkrecht auf ein Glan-Thomson Prisma, welches anhand eines Goniometer drehbar gelagert ist.
Sobald der Lichtstrahl aus dem Prisma austritt, ist dieser linear polarisiert. Mit Hilfe des Goniometers lässt sich 
der Polarisationswinkel ablesen. 

Unsere scheibenförmige Probe befindet sich in einem Elektromagneten, welcher durch ein Konstantstromgerät gespeist wird.
So ist die Probe mit einem zeitlich konstanten Magnetfeld durchsetzt.
Das linear polariserte Licht durchquert den Elektromagenten und die verwendete Probe.
Im Anschluss trifft der Lichtstrahl auf einen Interferenzfilter und wird so monchromatisert.
Das nahezu monochromatische Licht fällt auf einen zweiten Glan-Thomson Prisma und wird demnach
zu einen ordentlichen und außerordentlichen Lichtstrahl geteilt. Die Lichtstrahlen sind orthogonal zueinander linear polarisiert.
Die Strahlen treffen darauf jeweils auf einen Photowiderstand.

Beide Photowiderstände bestehen aus PbS und deren spektrale Empfindlichkeit befindet sich vom sichtbaren Bereich bis 
ins nahe Infrarot. Die Innenwiderstände sind über mehrere Zehnerpotzen propotional zur einfallenden
Lichtintensität. Ein Photowiderstand befindet sich jeweils in einem geerdeten Schaltkreis, welcher über eine Spannungsquelle 
und einen Vorwiderstand verfügt. Wenn nun ein Gleichstrom ausgehend von der Spannungsquelle durch den Vorwiderstand geschickt 
wird, kann der entstehende Spannungsabfall gemessen werden. Dadurch, dass das Licht gepulst durch die Photowiderstände läuft,
ist die am Photowiderstand abfallende Spannung auch eine Wechselspannung.

Die Signalspannung der Photowiderstände wird auf die beiden Eingänge des 
Differenzverstärkers gegeben. Die Ausgangsspannung ist demnach proportional zur Differenz der Eingangsspannungen.
Wenn beide Signale in ihrer Phase und ihrem Betrag übereinstimmen, verschwindet das Signal.
Zusätzlich wird ein Selektivverstärker eingebaut, welcher auf die Frequenz der Licht-Zerhackers eingestellt ist.
Schlussendlich wird das Signal des Selektivverstärkers an ein Oszillograph angeschlossen.

