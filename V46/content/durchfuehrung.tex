\section{Durchführung}
\label{sec:durchführung}

Im Folgeneden wird die Durchführung des Versuches beschrieben.

\subsection{Justierung des Versuchsaufbaus}
\label{sec:Justierung}

Zunächst werden Probe und Interferenzfilter aus der Apperatur entfernt. Die Halogenlampe wird bei einer
Spannung bis ungefähr \qty{11}{\volt} betrieben. Um sicherzustellen, dass die Lichtstrahlen korrekt auf die 
Photowiderstände abbilden, werden die Gehäuse der Photowiderstände abgebaut.
Weiter wird überprüft, ob das einfallende Licht senkrecht auf den Polarisator trifft.
Anschließend ist festzustellen, ob der polarisierte Lichtstrahl vollständig auf die Probe trifft.
Es wird nun geschaut, ob die beiden Lichtstrahlen jeweils mittig auf die Photowiderstände abbilden.
Durch Drehen des Polarisators sollte es anschließend möglich sein, das Licht zwischen den beiden Photowiderstände
hin und her zu schalten. Dabei ist es wichtig darauf zu achten, dass ein guter Kontrast vorliegt.

Die Gehäuse der Photowiderstände werden im Folgenden wieder eingebaut.
Daraufhin wird der Licht-Zerhacker eingebaut und auf eine Frequenz von ungefähr
\qty{443}{\hertz} eingestellt. Nun wird die Mittenfrequenz des Selektivverstärkers auf den eingestellten Wert geregelt.
Zur Justierung wird ein Photowiderstand an den Input des Selektivverstärkers geschlossen und anschließend
wird ein Oszillograph an den Ausgang "Resonance" geschlossen. Es kann nun der Selektivverstärker auf
das maximale Ausgangssignal eingestellt werden.
Schlussendlich wird eine \qty{90}{\degree} Periodizität am Differenzverstärkers überprüft.

\subsection{Messung der Faraday Rotation}
\label{sec:Faraday Rotation}

Insgesamt wird die Faraday-Rotation für drei Proben vermessen. Dabei werden zwei n-dotierte Gallium Arsenid Proben 
verwendet und anschließen eine hochreine Probe des Gallium Arsenid.
Für alle drei Proben werden in Summe 9 unterschiedliche Interferenzfilter zwischen dem Bereich \qty{1.06}{\micro\meter} 
bis \qty{2.65}{\micro\meter} eingebaut. Um $\theta$ zu messen wird so vorgegangen, dass bei maximalem Feld, beispielsweise bei
positivem Feld, die Lichtintensität in beiden Strahlen auf den gleichen Wert eingeregelt wird, sodass die Spannung am 
Differenzverstärker minimal wird.

Es werden Messwerte für die positive und die negative Polung notiert. Diese bennent man daher als $\theta_1$ und als $\theta_2$.

\subsection{Messung der magnetischen Kraftflussdichte}
\label{sec:Kraftflussdichte}

Das externe zeitlich konstanten Magnetfeld wird mit eine Hallsonde in Richtung des einfallenden Lichts vermessen.
Dafür werden zunächst der Photowiderstand und die Halterung der Interferenzfilters abgebaut, um anschließend
die Hallsonde zu plazieren. Diese wird per Augenmaß bis zur Mitte des Elektromagneten eingeführt. Das Konstantstromgerät
wird wie bisher verwendet auf einen maximalen Strom von ungefähr \qty{10}{\volt} eingestellt. In einem Abstand von 
\qty{1}{\milli\meter} wird die magnetische Flussdichte in einem Bereich von \qty{10}{\milli\meter} jeweils vor und hinter
der Mitte des Spaltes gemessen.
