\section[Theorie]{Theorie \textnormal{\cite{germanium}}}
\label{sec:theorie}

Zum Verständnis des hochreinen Germaniumdektektors werden die theorethschen Grundlagen dessen im Folgenden erläutert.

\subsection{Die Wechselwirkung von Strahlung mit Materie}
\label{sec:WW mit Materie}

Bei der Gamma-Strahlungs-Detektion werden Elektronen des Detektor-Matrials von den Gamma-Photonen angeregt und damit werden die 
Atome ionisiert. Daher bilden sich Elektronen-Loch-Paare. Die Primärenelektronen ionisieren wiederum weitere Atome des Detektor-Mediums und erzeugen somit weitere Elektron Loch Paare.
Die Anzahl der Elektronen-Loch Paare ist direkt proportional zur Energie des Elektrons aus der primären
Wechselwirkung. Da der Absorbtionskoeffizient für Gamma-Strahlung bei Gasen sehr niedrig ist, 
werden Gamma-Strahlen-Detektoren aus Festkörpern gebaut. Das Matrial des Detektors muss so gewählt werden, dass
die Anzahl Elektronen-Loch-Paare gesammelt und als elektrisches Signal wiedergegeben werden kann.
Zusätzlich ist der Grad der Interaktion von Gamma-Strahlung mit Materie abhängig von der Energie der Strahlung.
In Abbildung \ref{fig:koeffizient} ist der Dämpfungskoeffizient von Germanium, welcher die Reduzierung der Strahlungsintensität
bei bestimmter Energie verursacht durch den Absorber misst, gegen die Gamma-Strahlen Energie aufgetragen.

\begin{figure}[H]
    \centering
    \includegraphics[width=0.8\textwidth]{content/grafik/dämpfungskoeffizientGermanium.jpg}
    \caption{ \cite{gamma_ray}}
    \label{fig:koeffizient}
\end{figure}

Die totale Kurve setzt sich aus verschiedenen Komponenten zusammen: Der Photoeffekt, der Compton Effekt und die Paarerzeugung.
Im Folgenden werden die einzelnen Komponenten erläutert.

\subsubsection{Der Photoeffekt}
\label{sec:photoeffekt}

Der Photoeffekt beschreibt den Prozess bei dem ein ein Gamma-Quant mit einem Hüllenelektron wechselwirkt.
Das Photon wird absorbiert und das Elektron wird emittiert.

\subsubsection{Der Compton Effekt}
\label{sec:compton}

\subsubsection{Die Paarerzeugung}
\label{paarerzeugung}