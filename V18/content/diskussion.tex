\section{Diskussion}
\label{sec:diskussion}

Insgesamt zeigt sich bei der Auswertung der erwartete und entscheidende Vorteil hochreiner Germanium-Halbleiterdetektoren: Die
aus den Kanälen übersetzte Energieskala kann Strukturen sehr genau auflösen. Daher fallen die in der Literatur gegebenen Peak-Werte
teils bis in den Bereich moderner Messunsicherheiten von einigen \unit{\eV} mit den Messergebnissen zusammen.

Modellierung von Energiezugehörigkeit und Effizienz sind ebenfalls zufriedenstellend. Bei der Energiekalibration wird der
quasi perfekt lineare Zusammenhang des MCA deutlich. Die Nachweiswahrscheinlichkeit ist mit guter Übereinstimmung durch
den gewählten Exponentialansatz beschrieben. Hierbei fällt auf, dass die Effizienz für steigende Energien monoton abfällt.
Bei Werten ab etwa \qty{100}{\kilo\eV} ist dies plausibel, unterhalb dieser Grenze nimmt die Nachweiswahrscheinlichkeit
rapide gegen geringere Energien ab. Der Fit darf also nicht als global gültige Korrektur verwendet werden.

In diesem Zusammenhang stellt sich auch die Frage, nach der Anwendungsreihenfolge und der Verstärkung statistischer Schwankungen
durch Division mit $Q(E)$: So kann durch die Korrektur ein Bias in die Daten eingeführt werden, der anschließende Fits
verfälscht. Für höhere Energien werden Ausreißer zudem um zwei Größenordnungen verstärkt. Bei ausreichender Statistik
sollten diese Einflüsse zwar signifikant oberhalb der Ungenauigkeit der Kanal-Energie-Beziehung liegen, aber dennoch im Vergleich
zur vorherrschenden Streuung vernachlässigbar sein.

Auch das monochromatische Gammaspektrum erfüllt qualitativ die Erwartung: Photopeak und Comptonkontinuum sind klar identifizierbar,
Rückstreupeak und Comptonkante heben sich deutlich ab. In der genaueren Betrachtung fallen jedoch Probleme auf.

Der gaußförmige Fit um den Vollenergiepeak ist zwar mit dem geforderten FWTM zu FWHM Verhältnis vereinbar, scheint jedoch relativ
zum Datenverlauf geneigt zu sein. Diese asymmetrische Komponente ist kaum relevant für die Bestimmung des Gesamtinhalts, zeigt
aber ausreichende Signifikanz um über ihren Ursprung zu spekulieren. Eine mögliche Quelle liegt in der Effizienzkorrektur, wobei diese
eigentlich eine umgekehrte Schiefe verursachen müsste, bei der Werte zu höheren Energien nach oben tendieren. Möglicherweise handelt
es sich um eine Überlagerung mit einem schwächeren Peak aus einer Verunreinigung, die knapp unterhalb des Cäsium-Peaks liegt. Weitere
äußere Einflüsse können ebenso nicht ausgeschlossen werden, da die Bleikammer apparaturbedingt nicht völlig abgeschlossen ist.

Auf ähnliche Art und Weise wir der Comptonbereich beeinflusst. Die theoretisch klar definierten Spitzen sind in der realen
Verteilung gedämpft und verrauscht. Unter Berücksichtigung dieser Tatsache passen die aus der Theorie errechneten Werte von
Comptonkante und Rückstreupeak gut zu den Daten. Nach Integration ist außerdem eindeutig, dass der gemessene Inhalt circa
eine Größenordnung unterhalb der Erwartung liegt. Diese zunächst verblüffende Tatsache lässt sich mindestens teilweise durch
Detektorlimits erklären. So wird das Kontinuum im niederenergetischen Bereich abgeschwächt oder sogar abgeschnitten. Durch
stochastische Streuuprozesse im Behälter verzerrt die Verteilung zusätzlich bis zum Photopeak, sodass dort zusätzliche Signale
aufgezeichnet werden. Ein hinreichend flacher Rand sollte durch den Background-Fit kompensiert werden. Es ist letztlich dennoch
anzunehmen, dass der reale Inhalt des Photopeaks geringer und der des Comptonkontinuums größer ist, als hier geschätzt wird.

Zur Aktivitätsbestimmung treten die zuvor beschriebenen Störfaktoren in Kombination auf. Besonders deutlich ist die Nachweisgrenze für
niedrige Peak-Energien in Richtung harter Röntgenstrahlung zu erkennen. Die übrigen Werte widersprechen sich kaum und lassen somit auf
eine in diesem Energiebereich zuverlässige Messung schließen.

Zuletzt ist die hohe Konfidenz zur Identifikation der unbekannten Probe als Uranerz wieder mit der hohen Detektorauflösung
begründet, die eine eindeutige Zuordnung der Gammalinien erlaubt. 

Fehlereinflüsse wie das intrinsische Rauschen von Detektor und Ladungsverstärker können durch die Flüssigstickstoffkühlung und die
ebenfalls dadurch ermöglichte angelegte Hochspannung kontrollieren.
