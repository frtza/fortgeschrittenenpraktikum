\section{Durchführung}
\label{sec:durchführung}

Im ersten Abschnitt des Versuches soll die Energiekalibration und die Vollenergienachweiswahrscheinlichkeit
bestimmt werden. Dafür wird das Spektrum von kalibriertem Eu-152 aufgenommen.

Anschließend wird das Spektrum von Cs-137 gemessen. Anhand dessen soll das Spektrum und die Aktivität bestimmt werden.

Darauf folgt die Messung von Ba-133, um schließlich ebenfalls die Aktivität zu berechnen.

Zum Schluss wird das Gammaspektrum eines unbekannten Strahlers aufgenommen.
Mittels der gemessenen Daten soll dieser Strahler identifiziert werden, also dem aktiven 
Nuklid zugeordent werden.

Es wird die Länge des Abstandshalters mit einer Schieblehre vermessen.