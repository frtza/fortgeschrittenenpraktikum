\section{Procedure}
\label{sec:procedure}

Initially, the individual components need to be positioned and fastened on the breadboard. This process includes the
vertical and horizontal alignment of the laser grating, the CCD camera as well as the photodiodes.
Due to the laser beam being invisible to the human eye, the provided IR sensitive viewing card is used to check for centered
and maximally orthogonal incidence. After completing this first part, the lasing threshhold is evaluated by slowly adjusting the
current and checking for a change in the light pattern. Once this has been observed and the corresponding value has been noted down,
the resonant configuration for the rubidium fluorescence is searched for by iteratively changing both the grating angle and the
laser current until a flashing line becomes visible on the display. Finally, while ensuring no further changes to the setup, the
main focus of the measuring process is proceeded with. To perform saturated absorption spectroscopy, the contributing gain components
from the internal cavity as well as from the external cavity with grating feedback need to be modulated in a way such that
no mode hops occur during the scanning process over all rubidium lines of interest. To achieve this, the piezo stack and the laser
cavity are adressed simultaniously with a ramp current, changing the diffraction angle via piezoelectric expansion or contraction and
shifting the internal mode maximum. By tuning current and DC offset parameters, this method allows for comparably large continuous
sweeping intervals. Lastly, to remove the sloping background from the obtained signal, the photocurrent from the split beam is used
with some manually set coefficient to subtract from the waves altered from passing through the rubidium cell.
