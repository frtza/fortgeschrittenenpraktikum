\section{Theory}
\label{sec:theory}

The following section explains the theoretical principles of a diode laser.

\subsection{Historical Background}
\label{sec:Historical Background}

Before semiconductor lasers were invented, physicists used tunable 'dye' lasers.
This worked by the use of a chemical dye as the active medium, i.e the material which produces the laser emission.
A fixed-frequency 'pump'-laser is used to create a population inversion. Each individual dye will lase over a limited wavelenght range.
This means with different dyes it is possible to generate a tunable lasers at basically all near-infrared wavelenghts.
Dye Lasers have some disadvantages. They are very large, with high costs of purchase and operation.

The situation has changed due to the development of the diode laser. These lasers are inexpensive,easy to operate and produce high power.

\subsection{Diode laser}
\label{sec:Diode laser}

