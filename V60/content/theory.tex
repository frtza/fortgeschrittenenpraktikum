\section{Theory}
\label{sec:theory}

The following section explains the theoretical principles of a diode laser.

\subsection{Historical Background}
\label{sec:Historical Background}

Before semiconductor lasers were invented, physicists used tunable 'dye' lasers.
This worked by the use of a chemical dye as the active medium, i.e the material which produces the laser emission.
A fixed-frequency 'pump'-laser is used to create a population inversion. Each individual dye will lase over a limited wavelenght range.
This means with different dyes it is possible to generate a tunable lasers at basically all near-infrared wavelenghts.
Dye Lasers have some disadvantages. They are very large, with high costs of purchase and operation.

The situation has changed due to the development of the diode laser. These lasers are inexpensive,easy to operate and produce high power.

\subsection{Diode Laser}
\label{sec:Diode Laser}

The basics of how a diode laser works are explained below.

\subsubsection{Structure and mode of operation}
\label{Structure and mode of operation}

A diode laser is a laser whose light is generated with a laser diode, i.e. with semiconductor materials.
An important component is the diode chip which can be seen in figure \ref{fig:diodechip}.

\begin{figure}[H]
    \centering
    \includegraphics[width=0.8\textwidth]{content/graphics/laserdiodechip.jpg}
    \caption{Schematic view of a laser diode chip.} %\cite
    \label{fig:diodechip}
\end{figure}

The chip consists of a p-doped and a n-doped layer. The p-n junction between the layers is the active Medium.
An excitation current can be connected to the upper and lower ends of the diode to create electron-hole pairs which recombine in the active layer and
emitting light in the process. The current serves as a pump source for the population inversion which occurs in the laser medium.
The wavelength of the emitted light ist approximateley that of the band gab of the semiconductor material.
To create a cavity the long endings of the diode chip are impermeable whereby one end ist only semitransparent.
On this side the light comes out.
Standing waves are formed inside the cavity. The leaving light beam is elliptical and strongly divergent due to the  rectangular shape of the
exit apperature.

The structure of the entire laser is shown figure \ref{fig:configuration}

\begin{figure}[H]
    \centering
    \includegraphics[width=0.8\textwidth]{content/graphics/configuration.jpg}
    \caption{Schematic configuration of the diode laser system.} %\cite
    \label{fig:configuration}
\end{figure}