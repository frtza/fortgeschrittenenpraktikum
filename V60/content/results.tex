\section{Results}
\label{sec:results}

\begin{figure}
    \begin{subfigure}{0.48\textwidth}
        \centering
        \includegraphics[width=\linewidth]{content/measurement/led.jpg}
        \caption{LED regime at $I = \qty{34.4}{\milli\ampere}$.}
        \label{fig:pattern_led}
    \end{subfigure}
    \hfill
    \begin{subfigure}{0.48\textwidth}
        \centering
        \includegraphics[width=\linewidth]{content/measurement/laser.jpg}
        \caption{LASER regime at $I = \qty{34.6}{\milli\ampere}$.}
        \label{fig:pattern_laser}
    \end{subfigure}
    \caption{Comparison of light pattern slightly below and above the chosen threshold current $I = \qty{34.5}{\milli\ampere}$.
             Notice the diffuse reflection on the left versus the coarse granulation on the right. These distinct appearances
             are the results from random diffraction of incoherent or coherent waves respectively.}
    \label{fig:pattern}
\end{figure}

\begin{figure}
    \centering
    \includegraphics[width=0.48\linewidth]{content/measurement/fluorescence.jpg}
    \captionsetup{width=0.58\linewidth}
    \caption{Rubidium flourescence line along the laser beam as seen through the observation window in the vapor cell.}
    \label{fig:fluorescence}
\end{figure}

\begin{figure}
    \centering
    \includegraphics[width=0.72\linewidth]{content/measurement/ramp.jpg}
    \captionsetup{width=0.8\linewidth}
    \caption{Modulated signal without correction (top). Triggered ramp output for laser and piezo stack (bottom).
             As previously described, the expected linear proportionality between the two currents is clearly visible
             outside the absorption dips.}
    \label{fig:ramp}
\end{figure}

\begin{figure}
    \centering
    \includegraphics[width=0.72\linewidth]{content/measurement/spectrum.jpg}
    \captionsetup{width=0.8\linewidth}
    \caption{Spectrum after manual compensation via background subtraction for the scanning current contribution.}
    \label{fig:spectrum}
\end{figure}
