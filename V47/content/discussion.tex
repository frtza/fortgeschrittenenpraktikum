\section{Discussion}
\label{sec:discussion}

The experimental value for the Debye temperature is $\vartheta_{\text{exp}} = \qty{299.27+-0.19}{\kelvin}$.
The following results for the theoretical Debye temperature $ \vartheta_{\text{theo}} = \qty{332.208}{\kelvin}$.
It follows a relative deviation of the experimental value of $\num{9.91}\%$.
The deviation is within the acceptable range, as systematic errors may have occurred during the experiment.
First, heat convection never completely ruled out, which could have influenced the measurement.
Second, the sample was not completely insulated, which could have led to heat loss.

The theoretical molar heat capacity of copper is $C_{\text{theo}} = \qty{24.76}{\joule\per\mol\per\kelvin}$ at room temperature \cite{kupfer}.
The experimental value is $C_{\text{exp}} = \qty{24+-15}{\joule\per\mol\per\kelvin}$ at a temperature of $\vartheta = \qty{273.13+-0.26}{\kelvin}$.
It follows a relative deviation of the experimental value of $\num{0+-6} \%$.
The deviation is within the acceptable range.

The error could have arisen from the same facts as by the Debye temperature.
The asymptotic progression of the heat capacity for high temperatures can be recognized from the graph \ref{fig:heat_capacity_plot}.

In summary, it can be said that the error calculation has resulted in large uncertainties in the values.

Overall, the evaluation of the test was successful and the experimental values are within the acceptable range.