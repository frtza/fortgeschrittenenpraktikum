\section{Discussion}
\label{sec:discussion}

The experimental value for the Debye temperature is $\vartheta_{\text{exp}} = \qty{299.27+-0.19}{\kelvin}$.
The theoretical Debye temperature is calculated to be $\vartheta_{\text{theo}} = \qty{332.208}{\kelvin}$.
From this follows a relative deviation of the experimental value by $\num{9.91}\%$.
The deviation is within the acceptable range, as systematic errors may have occurred during the experiment.
First, heat convection can never be ruled out entirely, which could have influenced the measurement.
Second, the sample was not completely insulated, which could have led to heat losses.

The theoretical molar heat capacity of copper is $C_{\text{theo}} = \qty{24.76}{\joule\per\mol\per\kelvin}$ at room temperature \cite{kupfer}.
The experimental value is $C_{\text{exp}} = \qty{24+-15}{\joule\per\mol\per\kelvin}$ at a temperature of $T = \qty{273.13+-0.26}{\kelvin}$.
This closely matches the expectation with an uncertainty of $\num{6} \%$.
The error could have arisen from the same factors as the Debye temperature.

The asymptotic progression of the heat capacity for high temperatures can be recognized in graph \ref{fig:heat_capacity_plot}.

In summary, it can be said that the error calculation produces large uncertainties in the values.

Overall, the evaluation of the test was successful and the experimental values can be accepted with the given caveats.

\enlargethispage{\baselineskip}\newpage
