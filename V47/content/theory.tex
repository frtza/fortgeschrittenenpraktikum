\section[Theory]{Theory \textnormal{\cite{GrossMarx+2022}}}
\label{sec:theory}

In the description of condensed matter, tools from classical thermodynamics and statistical physics as well
as quantum mechanical considerations need to be applied. Throughout the following paragraphs, we explore some
of the more well known models in the subfield of thermal properties, specifically for the explanation of the
heat capacity in solids.

\subsection{Heat Capacity}

The heat capacity of any material is defined via
\begin{equation*}
	C \equiv \pfrac{\up{\Delta}Q}{\up{\Delta}T\:}
\end{equation*}
as the amount of heat $\up{\Delta}Q$ corresponding to a unit change $\up{\Delta}T$ in temperature. This is
clearly dependent on the amount of matter, so to allow for comparisons between different materials we can
normalize with volume $V$ or mass $M$ to obtain a specific heat capacity.

\subsubsection{Molar Heat}

In our case, we are interested in the heat capacity per number of particles making up a sample. This leads to
\begin{equation*}
	c \equiv \pfrac{\up{\Delta}Q}{n\up{\Delta}T\:}
\end{equation*}
for the molar heat capacity, where $n$ is the number of \unit{\mole} contained in the substance. To determine
$n$ one can use readily available values of the molar mass $m$ or molar volume $v$ for the given
experimental parameters.

\subsubsection{Measurement}

During the measuring process, it is necessary to constrain certain parameters. To grasp the following relations,
we examine the first and second laws of thermodynamics
\begin{equation*}
	\up{d} U = \up{\delta} Q - \up{\delta} W = T \up{d} S - P \up{d} V
\end{equation*}
where $U$ is the internal energy, $Q$ and $W$ are heat and work, $T$ stands for temperature, $S$ for entropy,
$P$ for pressure and $V$ for volume. The symbols $\up{d}$ and $\up{\delta}$ denote exact and inexact differentials,
respectively. \newpage

From this relation, we identify $\up{\delta} Q = T \up{d} S$ and thereby
\begin{equation*}
	C = T \: \pfrac{\del S}{\del T\:}
\end{equation*}
when translating the defining statement to infinitesimal notation.

\paragraph{Isochoric Case}

At fixed volume, all units of heat result in temperature changes exclusively. The isochoric heat capacity can be written as
\begin{equation*}
	C_V = T \left.\pfrac{\del S}{\del T\:}\right|_V
\end{equation*}
and constitutes an interesting quantity to study the behaviour of materials due to the isolation of otherwise connected effects.

\paragraph{Isobaric Case}

When the pressure is kept constant instead, some of the heat exchange also results in expansion or contraction of the
sample and therefore reduces the total temperature difference. Intuition therefore tells us that $C_P > C_V$ where we write
\begin{equation*}
	C_P = T \left.\pfrac{\del S}{\del T\:}\right|_P
\end{equation*}
for the isobaric heat capacity. This situation is more easily realizable in experiments but comes at the cost of no longer
separating different mechanisms affecting the substance.

\paragraph{Connection}

In order to convert between $C_P$ and $C_V$ we need some generally applicable relationship between the two quantities.
We establish the differential form for entropy
\begin{equation*}
	\up{d}S = \left.\pfrac{\del S}{\del T\:}\right|_V \!\up{d}T + \left.\pfrac{\del S}{\del V\:}\right|_T \!\up{d}V
\end{equation*}
in terms of volume as an extensive and temperature as an intensive property. We obtain
\begin{equation*}
	\left.\pfrac{\del S}{\del T\:}\right|_P \!= \left.\pfrac{\del S}{\del T\:}\right|_V \!+
	\left.\pfrac{\del S}{\del V\:}\right|_T \left.\pfrac{\del V}{\del T\:}\right|_P
\end{equation*}
by differentiating, allowing us to reformulate our desired quantites
\begin{equation*}
	C_P - C_V = T\left(\left.\pfrac{\del S}{\del T\:}\right|_P \!- \left.\pfrac{\del S}{\del T\:}\right|_V \:\right) =
	T\left.\pfrac{\del S}{\del V\:}\right|_T \left.\pfrac{\del V}{\del T\:}\right|_P
\end{equation*}
as the difference between them. \newpage

A combination of Jacobian coordinate transformations and Maxwell relations yields
\begin{equation*}
	C_P - C_V = \alpha_V^2 \, \kappa_T \, TV
\end{equation*}
with the isothermal bulk modulus
\begin{equation*}
	\kappa_T = -V\left.\pfrac{\del P\:}{\del V\:}\right|_T
\end{equation*}
and the volumetric thermal expansion coefficient
\begin{equation*}
	\alpha_V = \pfrac{1}{V\:} \left.\pfrac{\del V\:}{\del T\:}\right|_P
\end{equation*}
of the material. For isotropic materials, the last term can be expressed as $\alpha_V = 3\,\alpha_{\!L}$ via the linear analogue
\begin{equation*}
	\alpha_{\!L} = \pfrac{1}{L\:} \left.\pfrac{\del L\:}{\del T\:}\right|_P
\end{equation*}
which gives
\begin{equation*}
	C_P - C_V = 9 \, \alpha_{\!L}^2 \, \kappa_T \, TV
\end{equation*}
as the final result.

\subsection{Models}

\subsubsection{Classical Physics}

\subsubsection{Quantum Mechanics}

\paragraph{Phonons}

\paragraph{Einstein}

\paragraph{Debye}

\paragraph{Electrons}
