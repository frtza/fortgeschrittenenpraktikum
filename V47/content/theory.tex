\section[Theory]{Theory \textnormal{\cite{GrossMarx+2022}}}
\label{sec:theory}

In the description of condensed matter, tools from classical thermodynamics and statistical physics as well
as quantum mechanical considerations need to be applied. Throughout the following paragraphs, we explore some
of the more well known models in the subfield of thermal properties, specifically for the explanation of the
heat capacity in solids.

\subsection{Heat Capacity}

The heat capacity of any material is defined via
\begin{equation*}
	C \equiv \pfrac{\up{\Delta}Q}{\up{\Delta}T\:}
\end{equation*}
as the amount of heat $\up{\Delta}Q$ corresponding to a unit change $\up{\Delta}T$ in temperature. This is
clearly dependent on the amount of matter, so to allow for comparisons between different materials we can
normalize with volume $V$ or mass $M$ to obtain a specific heat capacity.

\subsubsection{Molar Heat}

In our case, we are interested in the heat capacity per number of particles making up a sample. This leads to
\begin{equation*}
	c \equiv \pfrac{\up{\Delta}Q}{n\up{\Delta}T\:}
\end{equation*}
for the molar heat capacity, where $n$ is the number of \unit{\mole} contained in the substance. To determine
$n$ one can use readily available values of the molar mass $m$ or molar volume $v$ for the given
experimental parameters.

\subsubsection{Isobaric Case}

\subsubsection{Isochoric Case}

\subsection{Models}

\subsubsection{Classical Physics}

\subsubsection{Quantum Mechanics}

\paragraph{Phonons}

\paragraph{Einstein}

\paragraph{Debye}

\paragraph{Electrons}
