\section{Procedure}
\label{sec:procedure}

Initially, the probe needs to be cooled down. First, we evacuate the recipient and replace the air with
helium gas. Due to its low boiling point of \qty{4}{\kelvin} the new medium remains gaseous even
at temperatures around \qty{77}{\kelvin} where air and nitrogen would liquify. This allows us to
achieve lower pressures by again removing all gases from the recipient after the lowest temperature
is reached in order to prevent heat losses via conduction, convection, evaporation or radiation.
Now we proceed with taking actual measurements. To this end, the probe and cylinder are supplied
energy via electrical power, minimzing the energy gradient and thereby increasing accuracy.
We record intervals of temperature $\Delta T$ and time $\Delta t$ as well as voltage $U$ and
current $I$ to determine $\Delta Q = UI \Delta t$ and from this the isobaric heat capacity.
Finally, we convert to the isochoric case and discuss some important observations.
\enlargethispage{\baselineskip}\newpage
