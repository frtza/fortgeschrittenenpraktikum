\section{Results}
\label{sec:results}

In the following, the molar heat capacity and then the Debye temperature are determined

\subsection{Parameters}
\label{sec:parameters}

The copper sample in this experiment has a mass of $m = \qty{0.342}{\kilo\gram}$ \textnormal{\cite{molar_heat}} and a density of
$ \rho = \qty{8930}{\kilo\gram \per \meter^3 }$ \textnormal{\cite{kupfer}}.
The table \ref{tab:parameters} shows the molar volume and the compression modul of copper.
%erstelle tabelle
\begin{table}[H]
	\centering
	\caption{Parameters of the copper sample}
	\label{tab:parameters}
	\begin{tabular}{c c }
		\toprule
		Parameter & Value \\
		\midrule
		$V_m$ & $\qty{7.11e-6}{\meter^3 \per \mol}$ \cite{chemie_kupfer}\\
		$\kappa$ & $\qty{140} {\giga\pascal}$ \cite{perioden_kupfer}\\
		\bottomrule
	\end{tabular}
\end{table}

The molar mass, the number of moles and the number of particles can then be calculated.
The following applies to the molar mass
\begin{equation*}
	M = V_m \cdot \rho = \qty{0.0634}{\kilo\gram \per \mol}.
\end{equation*}

The number of moles is calculated as follows
\begin{equation*}
	n = \frac{m}{M} = \num{5.103}
\end{equation*}

The result for the number of particles is
\begin{equation*}
	N = n \cdot N_A = \num{3.07e24}.
\end{equation*}

The volume of the sample is calculated as follows
\begin{equation*}
	V = \frac{m}{\rho} = \qty{3.628e-5}{\meter^3}.
\end{equation*}

The transverse velocity of sound and the longitudinal velocity of sound are also required for further calculations.
These are $v_t = \qty{2260}{\meter \per \second}$ and $v_l = \qty{4700}{\meter \per \second}$ \cite{molar_heat}.

\subsection{Theoretical Debye Temperature}
\label{sec:theoretical_debye_temperature}

The Debye temperature is calculated using the formula \ref{eqn:debye_temperature}.
The Debye frequency is determined by the formula \ref{eqn:debye_frequency}.
The result for the Debye temperature with the parameters given in section {sec:parameters} is
\begin{equation*}
	\vartheta_D = \qty{332.208}{\kelvin}.
\end{equation*}

\subsection{Calculation of the Molar Heat Capacity $C_p$}
\label{sec:calculation_of_the_molar_heat_capacity_cp}

The molar heat capacity $C_p$ can be calculated using the equation
\begin{equation}
	C_p = \frac{1}{n} \cdot \frac{E}{\Delta T}.
\end{equation}
Here the energy is defined as follows
\begin{equation}
	E = I \cdot U \cdot \Delta t.
\end{equation}

The measured values as well as the calculated results are shown in the table

\subsection{Calculation of the Molar Heat Capacity $C_v$}
\label{sec:calculation_of_the_molar_heat_capacity_cv}

The molar heat capacity $C_v$ can be calculated using the equation \ref{eqn:heat_capacity_difference}.
The values of the expansion coefficient can be read from the source \cite{molar_heat}.
The heat capacities, the temperatures and the expansion coefficient are shown in Figure .

\subsection{Experimental Debye Temperature}
\label{sec:experimental_debye_temperature}
