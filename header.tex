\documentclass[
	paper=a4,				% set din paper format
	parskip=half,			% vertically separate paragraphs
	bibliography=totoc,     % literature in table of contents
	captions=tableheading,  % table heading
	titlepage=firstiscover, % titlepage is cover
]{scrartcl}

% improve float package
\usepackage{scrhack}
% center overwide objects
\usepackage{adjustbox}

% warn if recompile is necessary
\usepackage[aux]{rerunfilecheck}

% essential math commands
\usepackage{amsmath}
% many math symbols
\usepackage{amssymb}
% extensions for amsmath
\usepackage{mathtools}

% font settings
\usepackage{fontspec} % loads Latin Modern Fonts automatically, alternatives: Libertinus Serif/Sans/Mono

% recalculate page layout after setting differing fonts
\recalctypearea{}

% german language settings
\usepackage[ngerman]{babel}


\usepackage[
	math-style=ISO,		% ┐
 	bold-style=ISO,		% │
	sans-style=italic,	% │ follow iso standard
	nabla=upright,		% │
	partial=upright,	% ┘
	warnings-off={				% ┐
		mathtools-colon,		% │ remove redundant warnings
		mathtools-overbracket,	% ┘
	},
]{unicode-math}

% traditional math fonts
\setmathfont{Latin Modern Math} % alternative: Libertinus Math
\setmathfont[range=\mathbb]{texgyrepagella-math.otf}
% \setmathfont{XITS Math}[range={scr, bfscr}]
% \setmathfont{XITS Math}[range={cal, bfcal}, StylisticSet=1]

% numbers and units num, unit, qty, ang
\usepackage[
	locale=DE,						% german settings
	separate-uncertainty=true,		% always use pm for uncertainties
	per-mode=reciprocal,			% always use negative exponent
%	per-mode=symbol-or-fraction,	% symbol in inline math, fraction in display math
]{siunitx}

% chemical formulas
\usepackage[
	version=4,
	math-greek=default,	% ┐ allow functioning with unicode-math
	text-greek=default,	% ┘
]{mhchem}

% corrected quotation marks
\usepackage[autostyle]{csquotes}

% additional fraction type sfrac to frac, dfrac, tfrac, cfrac
\usepackage{xfrac}

% default placement for floats here, top, bottom
\usepackage{float}
\floatplacement{figure}{htbp}
\floatplacement{table}{htbp}

% control float behaviour
\usepackage[
	section,	% force floats to remain inside section
	below,		% allow placement below section on same page
]{placeins}

% rotate pages to landscape orientation for wide tables
\usepackage{pdflscape}

% improved captions
\usepackage[
	labelfont=bf,			% produce bold label in caption
	font=small,				% reduced fontsize relative to document
	width=0.9\textwidth,	% reduced caption width
]{caption}
% subfigure, subtable, subref
\usepackage{subcaption}

% include graphics
\usepackage{graphicx}

% draw pgf illustrations
\usepackage{tikz}
% draw pgf circuits
\usepackage[european]{circuitikz}

% create tables
\usepackage{booktabs}
\usepackage{multirow}

% edit enumeration parameters
\usepackage{enumitem}

% improved typeface
\usepackage{microtype}

% bibliography
\usepackage[
	backend=biber,
]{biblatex}
% source database
\addbibresource{lit.bib}
\addbibresource{programme.bib}

% hyperlinks inside document hyperref, href, eqref, ref
\usepackage[
	german,
	unicode,        % allow unicode in pdf attributes
	pdfusetitle,    % title, author, date as pdf attributes
	pdfcreator={},	% ┐ clean up pdf attributes
	pdfproducer={},	% ┘
	hidelinks,		% hide boxes around links
]{hyperref}
% extended bookmarks inside pdf
\usepackage{bookmark}

% separate words with hyphen
\usepackage[shortcuts]{extdash}

% increase bracket size for nesting
\delimitershortfall=-1pt

% input useful macros
% Hammerite, https://tex.stackexchange.com/a/257122

\usepackage{pgfmath, xparse}

\newlength{\MathStrutDepth}
\newlength{\MathStrutHeight}
\settoheight{\MathStrutHeight}{$\mathstrut$}
\settodepth{\MathStrutDepth}{$\mathstrut$}

\newlength{\NumeratorDepth}
\newlength{\DenominatorHeight}
\newlength{\DepthNegativeDifference}
\newlength{\HeightPositiveDifference}
\newlength{\NumeratorBaselineCorrection}
\newlength{\DenominatorBaselineCorrection}

\newlength{\AdditionalEVSFracVerticalSpacing}
\setlength{\AdditionalEVSFracVerticalSpacing}{0.05mm}

% Fraction with equal top-and-bottom vertical spacing around the bar.
% Suited only to simple fractions that do not appear near other fractions.
% When used alongside other fractions, numerator and denominator baselines
% might not be aligned, which might give ugly results.
% Additionally, the default line thickness for overlines and fractions is restored.
\NewDocumentCommand\pfrac{omom}{%
% 		\Umathfractionrule\displaystyle=0.4pt\relax
% 		\Umathoverbarrule\displaystyle=0.4pt\relax
% 		\Umathoverbarvgap\displaystyle=1.4pt\relax
% 		\Umathfractionrule\textstyle=0.4pt\relax
% 		\Umathoverbarrule\textstyle=0.4pt\relax
% 		\Umathoverbarvgap\textstyle=1.4pt\relax
% 		\Umathfractionrule\crampeddisplaystyle=0.4pt\relax
% 		\Umathoverbarrule\crampeddisplaystyle=0.4pt\relax
% 		\Umathoverbarvgap\crampeddisplaystyle=1.4pt\relax
% 		\Umathfractionrule\crampedtextstyle=0.4pt\relax
% 		\Umathoverbarrule\crampedtextstyle=0.4pt\relax
% 		\Umathoverbarvgap\crampedtextstyle=1.4pt\relax
    \IfValueTF{#1}%
              {\settodepth{\NumeratorDepth}{$#1$}}%
              {\settodepth{\NumeratorDepth}{$#2$}}%
    \IfValueTF{#3}%
              {\settoheight{\DenominatorHeight}{$#3$}}%
              {\settoheight{\DenominatorHeight}{$#4$}}%
    \pgfmathsetlength%
        {\DepthNegativeDifference}%
        {\NumeratorDepth - \MathStrutDepth}%
    \pgfmathsetlength%
        {\HeightPositiveDifference}%
        {\MathStrutHeight - \DenominatorHeight}%
    \pgfmathsetlength%
        {\NumeratorBaselineCorrection}%
        {\AdditionalEVSFracVerticalSpacing + \DepthNegativeDifference + \HeightPositiveDifference}%
    \pgfmathsetlength%
        {\DenominatorBaselineCorrection}%
        {-\AdditionalEVSFracVerticalSpacing}%
    \def\Numerator{\raisebox{\NumeratorBaselineCorrection}{$#2$}}%
    \def\Denominator{\raisebox{\DenominatorBaselineCorrection}{$#4$}}%
    \frac{\Numerator}{\Denominator}%
}
 % fraction type with equal v and h spacing pfrac

% tighter dotting in table of contents
\makeatletter
	\renewcommand\@dotsep{2}
	\renewcommand\@pnumwidth{0.75em}
\makeatother

% shortened commands
\def\rb{\raisebox}
\def\bm{\symbf}
\def\up{\symup}
\def\del{\partial}
\def\delt{\pfrac{\del}{\del t}}
\def\deltt{\pfrac{\del^2}{\del t^2}}
\def\delx{\pfrac{\del}{\del x}}
\def\delxx{\pfrac{\del^2}{\del x^2}}
\def\dely{\pfrac{\del}{\del y}}
\def\delyy{\pfrac{\del^2}{\del y^2}}
\def\delz{\pfrac{\del}{\del z}}
\def\delzz{\pfrac{\del^2}{\del z^2}}
\def\delr{\pfrac{\del}{\del r}}
\def\delrr{\pfrac{\del^2}{\del r^2}}
\def\delphi{\pfrac{\del}{\del \phi}}
\def\delphiphi{\pfrac{\del^2}{\del \phi^2}}
\def\deltheta{\pfrac{\del}{\del \theta}}
\def\delthetatheta{\pfrac{\del^2}{\del \theta^2}}
\def\delvphi{\pfrac{\del}{\del \varphi}}
\def\delvphivphi{\pfrac{\del^2}{\del \varphi^2}}
\def\delvtheta{\pfrac{\del}{\del \vartheta}}
\def\delvthetavtheta{\pfrac{\del^2}{\del \vartheta^2}}

\author{
	Fritz Agildere\\ % Name
	\href{mailto:fritz.agildere@udo.edu}{fritz.agildere@udo.edu} % Email
	\and
	Amelie Strathmann\\ % Name
	\href{mailto:amelie.strathmann@udo.edu}{amelie.strathmann@udo.edu} % Email
}
\publishers{TU Dortmund – Fakultät Physik}

