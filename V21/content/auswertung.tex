\section{Auswertung}
\label{sec:auswertung}

Im Folgenden werden die aufgenommenen Messdaten ausgewertet, um den Kernspin der Isotope zu bestimmen.
Dafür müssen zunächst die Landé-Faktoren der Isotope bestimmt werden und die vertikale Komponente des Erdmagnetfeldes.
Anschließend wird das Isotopenverhältnis der Rubidium-Isotope bestimmt und der quadratische Zeeman-Effekt untersucht.

\subsection{Magnetfeld der Erde}
\label{sec:magnetfeld-der-erde}

Die Vertikalkomponente des Erdmagnetfeldes hat aufgrund des horizontal verlaufenden Lichtstrahls einen Einfluss auf die Messung.
Daher wird diese durch ein vertikal verlaufendes Magnetfeld kompensiert und der Aufbau wird um die vertikale Achse in Nord-Süd Richtung gedreht,
sodass die horizontale Komponente parallel oder antiparallel zu dem horizontalen Magnetfeld verläuft.
Zur Bestimmung der Magnetfeldstärke des vertikal verlaufenden Magnetfeldes, welches aus einer Horizontalen- und einer Sweep-Spule besteht, werden die Feldstärken
beider berechnet und addiert.
Für die Magnetfeldstärken der Spulen im Zentrum gilt
\begin{equation}
    B(0) = \frac{8  \mu_0  N I}{\sqrt{125} A}  \, .
\end{equation}
Die Stromstärke ergibt sich dabei aus der gemessenen Spannung des Sweepanteils und der des Horizontalanteils.
Anhand des Ohmschen Gesetzes $U = R \cdot I$ können die Ströme errechnet werden.

\subsection{Bestimmung Lande-Faktor}
\label{sec:best-lande-faktoren}

% create table with this data



\subsection{Kernspin der Rubidium-Isotope}
\label{sec:Kernspin der Rubidium-Isotope}

\subsection{Isotopenverhältnis}
\label{sec:Isotopenverhältnis}

\subsection{Quadratischer Zeeman-Effekt}
\label{sec:quadratischer-zeeman-effekt}