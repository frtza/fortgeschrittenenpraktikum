\section{Auswertung}
\label{sec:auswertung}

Im Folgenden werden die aufgenommenen Messdaten ausgewertet, um den Kernspin der Isotope zu bestimmen.
Dafür müssen zunächst die Landé-Faktoren der Isotope bestimmt werden und die vertikale Komponente des Erdmagnetfeldes.
Anschließend wird das Isotopenverhältnis der Rubidium-Isotope bestimmt und der quadratische Zeeman-Effekt untersucht.

Für die Magnetfeldstärken der Spulen im Zentrum gilt
\begin{equation}
    B(0) = \frac{8  \mu_0  N I}{\sqrt{125} A}  \, .
\end{equation}
Die Stromstärke des Sweepanteils wird per Umdrehnungen gemessen.
Dieser Wert wird mit \qty{0.1}{\ampere} pro Umdrehung umgerechnet.
Für den horizontalen Anteil wurde die Spannung in \unit{\milli\volt} gemessen und kann umgerechnet werden in \unit{\milli\ampere},
indem die Werte der Spannungen verdoppelt werden.

\subsection{Magnetfeld der Erde}
\label{sec:magnetfeld-der-erde}

Die Vertikalkomponente des Erdmagnetfeldes hat aufgrund des horizontal verlaufenden Lichtstrahls einen Einfluss auf die Messung.
Daher wird diese durch ein vertikal verlaufendes Magnetfeld kompensiert und der Aufbau wird um die vertikale Achse in Nord-Süd Richtung gedreht,
sodass die horizontale Komponente parallel oder antiparallel zu dem horizontalen Magnetfeld verläuft.
Zur Berechnung der Feldstärke des Erdfeldes muss zunächst die magnetische Feldstärke des vertikalen Feldes gemessen werden.
Diese betrug in diesem Experiment \qty{0.23}{\ampere}.
Anschließend wird über den $y$-Achsenabschnitt des Graphens die horizontale Magnetfeldstärke bestimmt.



\subsection{Bestimmung Lande-Faktor}
\label{sec:best-lande-faktoren}

% create table with this data



\subsection{Kernspin der Rubidium-Isotope}
\label{sec:Kernspin der Rubidium-Isotope}

\subsection{Isotopenverhältnis}
\label{sec:Isotopenverhältnis}

\subsection{Quadratischer Zeeman-Effekt}
\label{sec:quadratischer-zeeman-effekt}