\section{Durchführung}
\label{sec:durchführung}

Zunächst ist der Aufbau zu justieren. Dazu werden solange iterativ Abstand, Ausrichtung und Höhe der Sammellinsen variiert, bis am
Galvanometer der Photozelle ein maximaler Ausschlag verzeichnet wird. Nachdem eine korrekte Positionierung der weiteren optischen
Filter überprüft ist, wird der gesamte Aufbau vor Umgebungsstrahlung abgeschirmt und unter Zuhilfenahme des beiliegenden Kompass
abgeschätzt in nordsüdlicher Orientierung ausgerichtet. Weiter gilt es die vertikale Komponente des Erdmagnetfeldes
auszugleichen. Da diese am Nullpeak der Intensität eine Verbreiterung auslöst, kann der Schritt als erfolgt betrachtet werden,
wenn dessen Breite minimal ist. Um eine bessere Auflösung zu erhalten, bietet es sich an die Sweepdauer zu verlängern, da sich
dann pro Intervall eine vollständigere Besetzungsinversion ausbilden kann. Sodann darf zur eigentlichen Messung übergegangen werden.
Dazu wird ein Funktionsgenerator als RF Schwingung dazugeschaltet und die Sägezahnfrequenz in Schritten von \qty{100}{\kilo\hertz} 
hochgepegelt. Bei jeder Frequenz wird der Sweepwert per Potentiometer auf die resonanten Minima der vorkommenden Rubidiumisotope
eingestellt, gegebenenfalls unter Verwendung des zu diesem Zwecke verbauten horizontalen Verschiebungsfeldes. Anhand der auf diese
Weise aufgenommenen Datenpunkte lassen sich die Horizontalkomponente und damit der Betrag des Erdmagnetfeldes, die Landé Faktoren
und die Kernspins beider Isotope sowie die relative Häufigkeit im verwendeten Gasgemisch ermitteln.
