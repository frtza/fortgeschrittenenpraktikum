\section{Diskussion}
\label{sec:diskussion}

Während des Prozesses der Auswertung ist aufgefallen, dass anstelle des zweiten Peaks erneut der erste vermessen wurde.
Um die diese Fehlerquelle in der Ausgleichsrechnung zu eliminieren, wurden die Werte des Rb-85 Peaks entfernt.

Der theoretische Wert des Erdmagnetfeldes in NRW beträgt \qty{48}{\micro\tesla} (\cite{erdmagnetfeldstärke}).
Die experimentell bestimmte Magnetfeldstärke der Erde hat den Wert \qty{39.39(15)}{\micro\tesla}.
Der experimentelle Wert weicht um \num{17.94(0.31)} \% nach unten von dem theoretischen Wert ab.
Dieser Fehler liegt im akzeptablen Bereich, da während des Experimentes systematische aufgetreten sein können.
Zunächst kann die genaue örtliche Lage der Versuchsdurchführung einen Einfluss auf den Wert haben, weil
der Aufbau sich in einem Gebäude aus Beton und Stahl befindet.
Dies kann das $B$-Feld abschirmen.
Zudem musste der Versuchsaufbau in Nord-Süd Richtung justiert werden, dies wurde per Augenmaß abgeschätzt.
Durch Einstellung des vertikalen Magnetfeldes sollte der vertikale Teil des Erdmagnetfeldes kompensiert werden.
Infolgedessen wurde dieses vertikale $B$-Feld anhand des Oszillografen eingestellt und abgelesen.
Der horizontale Anteil des Erdmagnetfeldes wurde anhand einer Linearen Ausgleichsrechnung für jeweils beide Isotope bestimmt.
Anschließend wurde der Mittelwert beider Datenpunkte berechnet.
Daraus folgt ein statistischer Fehler, welcher das experimentelle Ergebnis beeinflussen kann.

Rb-87 hat in der Theorie einen Landé-Faktor von $g_F^{87} = 1/2$ und der Landé-Faktor des anderen Isotops besitzt den Wert $g_F^{85} = 1/3$.
Bei dem Versuch wurden die experimentellen Werte $g_{87} =$ \num{0.487 +- 0.004}  und $g_{85} = $  \num{0.328 +- 0.001} berechnet.
Das Ergebnis des Rb-87 Isotops weicht um \num{2.6(0.07)} \% nach unten ab und bei dem Rb-85 weicht der experimentelle Wert um \num{1.65(0.016)} \% nach unten ab.
Daher konnte der Landé-Faktor experimentell sehr annehmbar bestimmt werden.
Kleinere Abweichungen können aus den statistischen Fehlern folgen, die währende der Fehlerrechnung entstehen.
Außerdem wurden die einzelnen Peaks händisch vermessen, was ebenfalls zur Fehlerquelle werden kann.

Die theoretischen Werte des Kernspins der Isotope sind $I_{85} = 5/2$ und $I_{87} = 3/2$.
Die experimentell bestimmten Werte sind $I_{85} = \num{2.511(1)}$ und $I_{87} = \num{1.520(5)}$.
Der experimentelle Wert des Rb-85 Isotops weicht um \num{0.44(0.04)} \% nach oben ab und der Wert des Rb-87 Isotops weicht um \num{1.3(0.4)} \% nach oben ab.
Die Abweichungen sind sehr gering und liegen im akzeptablen Bereich.
Diese Unterschiede können durch die statistischen Fehler, die während der Fehlerrechnung entstehen, erklärt werden.
Zusätzlich sollten immer die systematischen Fehler beachtet werden, die während des Versuches auftreten können.

Das Isotopenverhältnis in dem vorliegenden Versuch beträgt 1:2, wobei Rb-85 das häufigere Isotop ist.
In der Natur beträgt der Anteil von Rb-87 27.83 \% und der Anteil von Rb-85 72.17 \% (\cite{rubidium}).
Es kann vermutet werden, dass Rb-87 in dem Versuch angereichert wurde, um den entsprechenden Peak besser sichtbar zu machen.

Abschließend werden die Größenordnungen der Zeeman-Aufspaltungen mit der Hyperfeinstruktur Aufspaltung verglichen.
Die Zeeman-Aufspaltungen liegen im Bereich \unit{\nano\eV}.
Die Hyperfeinstruktur Aufspaltung ist um vier Größenordnungen größer.

Insgesamt konnte die Auswertung des Versuches erfolgreich durchgeführt werden und die experimentellen Werte liegen im akzeptablen Bereich.
